%-------------------------------------------------------------------------------
%	SECTION TITLE
%-------------------------------------------------------------------------------
\cvsection{Outreach \& Professional Development}


%-------------------------------------------------------------------------------
%	CONTENT
%-------------------------------------------------------------------------------
% \cvsubsection{Service and Outreach}
% \begin{cvhonors}
% %---------------------------------------------------------
%   \cvhonor
%     {Science Fair} % Event/Organization
%     {Volunteer Judge} % Position
%     {} % Location
%     {2018} % Date(s)
    
% %---------------------------------------------------------
%   \cvhonor
%     {Graduate Student Association} % Event/Organization
%     {Committee Member} % Position
%     {} % Location
%     {2016-2017} % Date(s)
    
% \end{cvhonors}

% \cvsubsection{Development}

% \begin{cvpubs}
%     \cvpub{\textbf{Workshop I Participated In}, and an explanation of this workshop, what I learned from it, and why I consider it to be professional development.}
% \end{cvpubs}
\cvsubsection{Visiting Period}\\
\begin{small} \color{black}
\textcolor{gray}{March 2024 - June 2024\\ 
\textit{Digital Signal Processing and Image Analysis (DSB) lab at the University of Oslo (UiO)}} \\
Collaboration with \linked{https://www.mn.uio.no/ifi/english/people/aca/adinr/}{Prof. Adín Ramírez Rivera} on the exploration of ViT latent spaces for Semantic Segmentation and Few Shot Segmentation. Specifically, I investigated the reuse of patch tokens from a ViT model pretrained on ImageNet for classification to perform Semantic Segmentation on ADE20K. This approach aimed to evaluate the generalization capabilities of ViTs and reduce training requirements when adapting to new datasets and tasks.
\end{small}


\vspace{3mm}
\cvsubsection{Tool Development}\\
\begin{small} \color{black}
\textcolor{gray}{2022 - 2023}\\ 
Developed a semiautomatic segmentation tool for grape labeling in RGB images, using a segmentation model trained on one grape species and deployed on another to accelerate annotation through knowledge transfer.\\
\linktesi{https://github.com/maxfehrentz/SEMI-AUTOMATIC-SEGMENTATION-TOOL}
\end{small}


% \vspace{0.7cm}
% \cvsubsection{Professional Memberships}\\
% \begin{small} \color{black}
% IEEE Student Membership
% \end{small}



% \cvsubsection{Tool Development}
% \begin{small} \color{black}

% %\textcolor{gray}{2022 - 2023}\\ 
% Participation in the development of a semiautomatic segmentation tool for RGB images
% \href{https://github.com/maxfehrentz/SEMI-AUTOMATIC-SEGMENTATION-TOOL}
% \end{small}



